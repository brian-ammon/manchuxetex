\documentclass{article}
\usepackage{fontspec}
\usepackage{booktabs}
\setmainfont{Brill}
\usepackage{manchuxetex}
\newfontfamily\manchufont{Mongolian Baiti}
\newfontfamily\chinesefont{STSong}
\begin{document}
\newcommand{\mncletter}[1]{\raisebox{.5em}{\rotatebox{-90}{\textmanchu{#1}}} & \texttt{#1}}
\newcommand{\textchinese}[1]{{\chinesefont #1}}
\begin{table}[htbp]
\caption{Transcriptions of Manchu}
\begin{center}
\begin{tabular}{clllll}
\toprule
mnc & mx & mdf & haen & xmhdcd & zakh\\
\midrule
\mncletter{a} & \multicolumn{3}{c}{a} & а\\
\mncletter{e} & \multicolumn{3}{c}{e} & э\\
\mncletter{i} & \multicolumn{3}{c}{i} & и (й)\\
\mncletter{o} & \multicolumn{3}{c}{o} & о\\
\mncletter{u} & \multicolumn{3}{c}{u} & у\\
\mncletter{U} & ū & ô & uu & ӯ\\
\mncletter{n} & \multicolumn{3}{c}{n} & н (нь)\\
\mncletter{k} & \multicolumn{3}{c}{k} & к (к̄) (къ)\\
\mncletter{g} & \multicolumn{3}{c}{g} & г (г̄)\\
\mncletter{h} & \multicolumn{3}{c}{h} & х (х̄)\\
\mncletter{b} & \multicolumn{3}{c}{b} & б (бъ)\\
\mncletter{p} & \multicolumn{3}{c}{p} & п\\
\mncletter{s} & \multicolumn{3}{c}{s} & с (съ)\\
\mncletter{S} & š & ś & sh & ш\\
\mncletter{t} & \multicolumn{3}{c}{t} & т (тъ)\\
\mncletter{d} & \multicolumn{3}{c}{d} & д\\
\mncletter{l} & \multicolumn{3}{c}{l} & л (лъ)\\
\mncletter{m} & \multicolumn{3}{c}{m} & м (мъ)\\
\mncletter{c} & \multicolumn{2}{c}{c} & ch & ч\\
\mncletter{j} & \multicolumn{2}{c}{j} & zh & чж (цзи)\\
\mncletter{y} & \multicolumn{3}{c}{y} & я ѣ іо ю ю̄\\
\mncletter{r} & \multicolumn{3}{c}{r} & р (ръ)\\
\mncletter{f} & \multicolumn{3}{c}{f} & ф\\
\mncletter{w} & \multicolumn{3}{c}{w} & в\\
\mncletter{k'} & \multicolumn{2}{c}{k’} & kk & к̄\\
\mncletter{g'} & \multicolumn{2}{c}{g’} & gg & г̄\\
\mncletter{h'} &  \multicolumn{2}{c}{h’} & hh & х̄\\
\mncletter{ts} & ts’ & z’ & c & ц\\
\mncletter{tsi} & ts & z’ & cy & цы\\
\mncletter{dz} & dz & z & z (-y) & цз (-ы)\\
\mncletter{Z} & ž & ź & r & ж\\
\mncletter{sy} & sy (\textchinese{四}) & s & sy & сы\\
\mncletter{cy} & c’y (\textchinese{勅}) & c’i & chy & чи\\
\mncletter{jy} & jy (\textchinese{智}) & j’i & zhy & чжи\\
\mncletter{ng} & \multicolumn{3}{c}{ng} & нг (нъ)\\

\bottomrule
\end{tabular}
\end{center}
\label{default}
\end{table}%


\end{document}